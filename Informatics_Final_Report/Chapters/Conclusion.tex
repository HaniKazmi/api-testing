\chapter{Conclusion and Future Work}

There are a myriad ways of testing an application, and automating all of them is difficult, arguably impossible. This thesis set out to tackle a subset of the tests currently manually conducted in software engineering companies, and I believe it has satisfactorily met this challenge. RATML provides an unambiguous, human and machine readable format for defining restful web APIs, while the Test Framework can transform the specification into a suite of varied tests. The tests can be executed against an implementation of the API, and the results reported to the user. By using popular, open source projects like YAML and ruby as a basis, the project has a good chance of catching on in the industry, and has already been adopted by Livedrive. 

The project is obviously not a complete system. Tests are limited to three major categories: Presence, schema and input/output. There are far more possible such as integration, component and functional. Furthermore, scheme test do not yet have a 100\% reliability rate, though I do not believe there are any other systems on the market which even attempt schema tests.

RATML and the Test Framework are centred around restful web APIs. While the Test Framework is generic enough that it could conceivably function with other restful services, RATML certainly can not and would need major expansion. These limitations mean that the project is currently only a viable solution for Livedrive and similar companies: it can not be used for other types of applications.

While the requirements for this thesis have been met, I will continue work on the Test Framework. Future extensions include adding support for Mocks and Fixtures. The Framework is currently limited to testing the API implementation as a whole, which is not always beneficial. A heuristic to determine components from the specification and target tests accordingly could help the Test Framework replace FitNesse.

RATML can also be improved. RAML contains constructs to define security mechanisms, and abstraction layers to allow the specification to be split over several files. These can be ported to RATML and implemented in the parser, allowing greater interoperability between RATNL and RAML.

Thanks to it's parallelized design, the application is highly scalable. While it is fast enough for it's current uses, there are several ways to gain more performance from it. The parser can be written in a lower-level, compiled language to speed up RATML parsing. While ruby has proven fast enough so far thanks to manipulating the YAML syntax tree directly, it has been at the cost of increasing memory usage. Writing the entire parser in C++ or Rust could help with this if it ever becomes a bottleneck.

The biggest roadblock for RATML and the Test Framework is adoption. Test suites do not live in isolation, and must consist of an ecosystem to be truly useful. Therefore, the entire project will be open sourced to encourage the community to contribute and improve it. However, it is competing with the already ingrained RAML and FitNesse. The Test Framework meets a specific niche, and that may help it be used alongside other products while the ecosystem builds.