\chapter{Requirements}

Below are a set of functional and non-functional requirements based upon which the system should be developed. Successful completion of this project should lead to a well defined way of specifing restful web apis, and using this specifcation to automatically test the implementation for bugs.

\section{User Requirements}

This section defines all the actions a user must be able to perform:

\begin{itemize}
\item Fully specify a restful web api, so that it can be clearly understood by developers and software architects.
\item Be able to define what 'correct' operation of the api results in.
\item Automatically run tests against an implementation of the web api based upon the specification.
\item Be able to define sample inputs for the restul web api.
\item Be able to define sample outputs for a given input.
\item Be able to check in the api returns the correct output for any given input.
\end{itemize}

\section{Functional Requirements}

This section defines the functionality the system must be able to accomplish.

\begin{itemize}
\item Parse the specification into a machine readable format.
\item Provide the user with a method to execute tests against a web api.
\item Report to the user how many tests failed.
\item Be able to send sample inputs to the restful api and ensure they are valid.
\item Be able to detect if an api implementation does not follow the specification.
\end{itemize}

\section{Non-Functional Requirements}

This section defines more subjective requirements which will help provide a high quality project. It is split into two parts: Specification and Automated testing.

\subsection{Specification}
\begin{itemize}
\item The specification language must be easily human readable.
\item The language must be able to suiccintly and unambiuously define a restful web api.
\item The language must be easily convertable to a format machines can process.
\item The language most be agnostic to any implementation details. It should not matter how the api is going to work, or which programming language it will be written in.
\end{itemize}

\subsection{Automated testing}
\begin{itemize}
\item The system should be reliable: it should return the same result if run with the same specification on the same implementatiom.
\item The system should be implementation agnostic: It should be able to run the tests no matter what language the api is written in.
\item The system should be able to run on as many common operating systems as feasibly possible.
\item The system should be performant, scaling linearly with the number of tests being run.
\item The system shpuld be easy to maintain, following development best practices and making good use of architecturel design patterns.
\item The system should be fully unit tested to provide assurences that the code is as correct as possible.
\end{itemize}